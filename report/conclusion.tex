We have successfully addressed the video action recognition task using the TimeSformer architecture, achieving strong performance on the HMDB dataset. 
Our best model achieves 90.3\% accuracy, 98.4\% top-5 accuracy and 81.6\% F1-score, significantly outperforming the R3D baseline (69.7\% accuracy) and the simple tuning of the TimeSformer model. 
Through systematic experimentation, we have demonstrated how TimeSformer model can be effectively adapted to the HMDB dataset.

Our experiments reveal several key insights.
The TimeSformer architecture proves particularly effective for this task due to its ability to capture both spatial and temporal relationships through self-attention.
The choice of sampling strategy significantly impacts model performance, with fixed sampling providing better accuracy while equidistant sampling offering more balanced performance across classes.
Data augmentation techniques help improve model robustness, though their effectiveness varies depending on the transformation type.

Future work could explore several promising directions. 
First, more thorough data exploration and preprocessing could be beneficial, particularly in terms of frame filtering and class balancing. 
The promising results from our initial frame filtering experiment suggest that a more comprehensive analysis of temporal redundancy could lead to further improvements. 
Second, the varying complexity of different action classes suggests that adaptive methods, which can adjust their processing based on action characteristics, might be more effective than uniform approaches. 
Finally, exploring architectures that can better handle the temporal dynamics of different action types could lead to more robust performance across all classes.
